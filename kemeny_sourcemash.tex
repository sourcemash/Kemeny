\documentclass[11pt]{article}
\title{\includegraphics[scale=0.65]{solologo} \\~\\ \includegraphics[scale=0.75]{logotext} \\~\\ \textbf{a web-based news aggregator \\ and categorization engine}}
\author{
	Alex Gerstein \\ Dartmouth College '15 \\ Computer Science \\ \texttt{alexander.s.gerstein.15@dartmouth.edu}
	\\ \\
	Scott Gladstone \\ Dartmouth College '15 \\ Computer Science \\ \texttt{scott.w.gladstone.15@dartmouth.edu}
	}
\date{\today}

\usepackage[margin=1.0in]{geometry}
\setlength{\parskip}{10pt plus 1pt minus 1pt}
\usepackage{fancyhdr}
\usepackage{enumerate}
\usepackage{multicol}
\usepackage{fixltx2e}
\usepackage{graphicx}
\graphicspath{ {images/} }
\usepackage[font=small,labelfont=bf]{caption}
\usepackage[superscript,biblabel]{cite}
\usepackage{url}

\newenvironment{Figure}
  {\par\medskip\noindent\minipage{\linewidth}}
  {\endminipage\par\medskip}

\begin{document}

\maketitle
%\begin{multicols}{2}

%%%%%%%%%%%%%%%%%%%%%%%%%%%%%%%%%%%
%%%% ********		   PROJECT OVERVIEW          ******** %%%%%
%%%%%%%%%%%%%%%%%%%%%%%%%%%%%%%%%%%

\pagebreak
\section{Project Overview}

Sourcemash is a web-based RSS feed aggregator that algorithmically categorizes articles for users to browse their news by category instead of by source. Typically, RSS feed readers only allow users to look through their news feed-by-feed. By categorizing articles and regrouping news in this way, users can find the topics that interest them faster and without re-reading similar stories from different sources. Sourcemash is ideal for news junkies who want the full picture on a story in a way that's easier to digest and consumes less of their time.

%%%%%%%%%%%%%%%%%%%%%%%%%%%%%%%%%%%
%%%% ********		  USER INSTRUCTIONS  		******** %%%%%
%%%%%%%%%%%%%%%%%%%%%%%%%%%%%%%%%%%

\section{User Instructions}

Sourcemash is a web application, located at \url{www.sourcemash.com}. The source code for Sourcemash is available at \url{https://github.com/sourcemash/Sourcemash}.

\subsection{Production}

Users are encouraged to use the production release of Sourcemash available at \url{www.sourcemash.com}. The website uses a Material Design framework\cite{Materialize} to create an easy-to-navigate, responsive user interface. The navigation menu, pinned to the left side of the page, is the primary tool used to access the site's core features and pages, which are explained in detail below:

\begin{itemize}
	\item \textbf{Browse} (\url{sourcemash.com/browse}): The Browse page contains an offering of feeds that users can subscribe to. If users do not see a feed that they wish to add, they may add their own using the form at the top of the page.
	\item \textbf{Categorizer} (\url{sourcemash.com/categorizer}): This page offers standalone access to the Sourcemash categorization engine. The categorizer allows users to input the URL of any online article and receive auto-generated categories based on our categorization algorithm.
	\item \textbf{Category or Feed View} (e.g. \url{sourcemash.com/feeds/1}): By clicking on a Category or Feed title under the tabs in the left-pinned navigation menu, users can view and interact with all items present in that category or feed.
	\item \textbf{Saved} (\url{sourcemash.com/saved}): Authenticated users can find the list of items that they have bookmarked here.
	\item \textbf{Profile} (\url{sourcemash.com/profile}): Authenticated users can manage their settings here. For example, Sourcemash offers a ``Show Suggested Content" feature that will display one additional article inside a category page from a feed that a user is not subscribed to. This helps introduce users to new feeds.
\end{itemize}

\subsection{Development}

Instructions for installation, execution, and testing are maintained at \url{https://github.com/sourcemash/Sourcemash/blob/master/README.md}.


%%%%%%%%%%%%%%%%%%%%%%%%%%%%%%%%%%%
%%%% ********      TECHNICAL DESCRIPTION 	******** %%%%%
%%%%%%%%%%%%%%%%%%%%%%%%%%%%%%%%%%%

\section{Technical Description}

Sourcemash was built to be used as an RSS feed aggregator, but at its core is a text categorization engine. We will break down the application's technical description into two components: the categorization algorithm and the RSS reader that incorporates the algorithm to allow users to parse through their news faster.

\subsection{Categorization Algorithm}

The categorization algorithm has two fundamental implementation requirements. First, because Sourcemash aims to group articles by category, the algorithm must generate categories with sufficient collisions or overlap among articles. Second, the algorithm needs to work quickly and efficiently in order to stay up-to-date with the latest news.

Satisfying the collision requirement required a strategy to normalize our category data. As an initial pass, this meant that capitalization, punctuation, and pluralization needs to be consistent for each instance of a given category. Upon researching state-of-the-art categorization techniques, we came across the CommunityCluster algorithm\cite{Grineva}, which uses Wikipedia lookups to generate possible keywords from an article's text. The Wikipedia API normalizes any query strings by removing any inconsistency with punctuation or capitalization.

For the efficiency requirement, we increased the speed of the CommunityCluster algorithm by pre-processing article data and running the algorithm on only the n-most-common ngrams in each feed article. This requires less lookups using the Wikipedia API, which is the biggest constraint on the algorithm's runtime.

The CommunityCluster algorithm requires a heuristic to determine the relatedness between any two Wikipedia articles. On Wikipedia, articles contain inline hyperlinks to related Wikipedia articles. The relatedness metric we used relies on the number of overlapping links between Wikipedia articles. By calculating the percent of hyperlinks that overlap between two Wikipedia articles, we can get a sense of their similarity.

The current implementation of the categorization algorithm works as follows:

\begin{enumerate}

\item Generate a bag-o-words for article $a$, overweighting counts for ngrams from the article title and longer phrases (e.g. bigrams, trigrams).
\item From the word counts generated in the bag-o-words, select the 20 most common ngrams as the keyword candidates $C$.
\item For each keyword candidate $c \in C$:
  \begin{enumerate}
  	\item Search Wikipedia for whether $c$ exists as a Wikipedia article $w$.
  	\begin{enumerate}
 		 \item If $w$ exists but is ambiguous (i.e. has a Wikipedia disambiguation page), map the keyword $c$ to all related Wikipedia articles linked to from the disambiguation page, denoted $W' = \{w'_{1}, w'_{2}, ...\}$.
  		\item Otherwise, map $c$ to the unambiguous Wikipedia article $w$.
  	\end{enumerate}
	\item Memoize the hyperlinks in $w$ (or in all $w' \in W'$) to quickly compute relatedness scores later.
  \end{enumerate}
\item For each $c$ that mapped to one unambiguous $w$, store $w$ in the list of assigned Wikipedia articles for the article $a$, denoted $A$.
\item For each $c$ that mapped to an ambiguous set $W'$, disambiguate $W'$ by summing the relatedness scores between each $w' \in W'$ and all articles $w \in A$ from Step 4. Append the $w'$ with the max total relatedness score to the list of assigned articles $A$.
\item Generate the graph $G$, where each $w \in A$ is a vertex and the edge weight $|e_{ww'}|$ is the relatedness score between articles $w$ and $w'$.
\item Perform the Louvain method for community detection\cite{Blondel} to isolate the communities in $G$.
\item Extract the densest communities' keywords to be used as the categories for article $a$.
\end{enumerate}

To test the categorizer on any online article, go to \url{sourcemash.com/categorizer}.

\subsection{RSS Reader}

The site is built as a RESTful API written in Flask, a Python web microframework, for the backend, and a single-page app built in Backbone.js for the frontend. A Redis server runs in the background to asynchronously schedule all emails and other worker tasks.

%%%%%%%%%%%%%%%%%%%%%%%%%%%%%%%%%%%
%%%% ********      		EVALUATION 			******** %%%%%
%%%%%%%%%%%%%%%%%%%%%%%%%%%%%%%%%%%

\section{Evaluation}

While other solutions exist to group and categorize related articles, only Sourcemash allows users to choose their own sources. Sourcemash is the first and only RSS feed reader that categorizes at the article-level to reorganize your news. This unique feature makes it stand out from competing RSS feed readers, such as Feedly.

\subsection{Accuracy}
While the categorization algorithm often finds accurate ``top'' categories for an article, it faces some issues with identifying poor secondary categories. For example, for a CNN article titled ``Deadliest outbreak of Ebola virus: What you need to know''\cite{Ebola}, accurate categories \texttt{Ebola Virus Epidemic in West Africa, Virus, Peace Corps, Liberia}, and \texttt{Sierra Leone} are returned along with innacurate categories \texttt{Gary Sick, Says}, and \texttt{Tornado outbreak}. These poor categorizations likely derive from incorrect disambiguations of words such as ``sick'' or ``natural disaster'' present in the article. 

As observed by Grineva, Grinev and Lizorkin\cite{Grineva}, the CommunityCluster algorithm is approximately 60-70\% effective at identifying accurate categories, but incorrectly disambiguated words sometimes pass the relatedness metric. To restrict the appearance of these incorrectly disambiguated words to users, Sourcemash performs additional processing on generated categories. For example, Sourcemash only displays categories that have a sufficient number of matching articles, which helps remove rarer, often incorrectly identified, categories. With Sourcemash, we have found a similar accuracy ratio in selected categories as Grineva's team (e.g. 5 of 8 [62.5\%] of the categories assigned to the CNN article\cite{Ebola} referenced above were considered to be ``accurate portrayals of the article'' by surveyed individuals). 

To maintain a level of specificity in the categories, we also ignore any categories that are ``nested'' or contained within other categories. For example, the categorization engine would assign ``Google Maps'' as a category over ``Google,'' if both are present in an article's set of assigned categories.

\subsection{Letter of Support: Devin Balkcom, Dartmouth College}

I'm writing in support of Alex Gerstein and Scott Gladstone's submission for the Kemeny Prize: ``Sourcemash.''  Scott and Alex did this work as their CS 98 project. The project was entirely conceived by Scott and Alex, and their work was entirely independent.

Sourcemash is a feed reader that collects articles from several sources, and automatically categorizes them.  The categorization allows users to identify and avoid reading redundant articles, and allows users to explore a particular event or topic more deeply. There does not appear to be any competing software that goes quite so far in this direction. 

The work represents an extremely impressive technical accomplishment. Scott and Alex worked 20-30 hours per week on this project (2-3 times what I asked) from the first day, with a very clear vision of what they wanted to build. Their software engineering skills are far beyond what I have seen in the past several years from other students -- in weekly meetings, they typically reported dozens of resolved issues, all carefully marked through Github tracking. Each new feature was built in parallel with automated tests. 

The project itself is interesting and usable. I do not read many feeds, and so do not have much experience with other feed readers, but the interface seems complete and professional. I'd use it. 

The automated categorization works surprisingly well. After initial attempts at using techniques based on word importance within the provided text, Alex and Scott abandoned this approach, and started to use clustering with external (Wikipedia) articles to find categorizations. There are occasional strange results, but the categorization is good enough to be genuinely useful.

In summary, this is truly excellent work, and I give this project my highest recommendation.


%%%%%%%%%%%%%%%%%%%%%%%%%%%%%%%%%%%
%%%% ********		ACKNOWLEDGEMENTS  	******** %%%%%
%%%%%%%%%%%%%%%%%%%%%%%%%%%%%%%%%%%

\subsection*{Acknowledgments}
Devin Balkcom, Professor of Computer Science at Dartmouth College, provided key guidance in the direction of Sourcemash's development throughout the duration of CS 98: Senior Design and Implementation Project. Michael Evans and Sravana Reddy, Neukom Fellows from the Neukom Institute at Dartmouth College, engaged in useful discussion with the authors and provided plentiful advice and feedback. Without their support, Sourcemash could not have achieved such success.

%\end{multicols}

\begin{thebibliography}{99}

\bibitem{Materialize}
  ``Materialize: A modern responsive front-end framework based on Material Design.'' Available at \url{http://materializecss.com/}. MIT, 2015.

\bibitem{Grineva}
  M. Grineva, M. Grinev and D. Lizorkin, ``Extracting key terms from noisy and multi-theme documents.'' \emph{Proceedings of the 18th international conference on World wide web}. ACM, 2009.

\bibitem{Blondel} Blondel, V.~D., Guillaume, J.-L., Lambiotte, R., \& Lefebvre, E.``Fast unfolding of communities in large networks.'' \emph{Journal of Statistical Mechanics: Theory and Experiment} 10(8), 2008.

\bibitem{Ebola} Cullinane, S. and Thompson, N. ``Deadliest outbreak of Ebola virus: What you need to know.'' \emph{CNN}. Available at \url{http://www.cnn.com/2014/03/27/world/ebola-virus-explainer/}, 2014.

\end{thebibliography}

\end{document}

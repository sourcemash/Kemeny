\documentclass[11pt]{article}
\title{\includegraphics[scale=0.65]{solologo} \\~\\ \includegraphics[scale=0.75]{logotext} \\~\\ Kemeny Prize Application 2015}
\author{
	Alex Gerstein \\ Dartmouth College '15 \\ Computer Science \\ \texttt{alexander.s.gerstein.15@dartmouth.edu}
	\\ \\
	Scott Gladstone \\ Dartmouth College '15 \\ Computer Science \\ \texttt{scott.w.gladstone.15@dartmouth.edu}
	}
\date{\today}

\usepackage[margin=1.0in]{geometry}
\setlength{\parskip}{10pt plus 1pt minus 1pt}
\usepackage{fancyhdr}
\usepackage{enumerate}
\usepackage{multicol}
\usepackage{fixltx2e}
\usepackage{graphicx}
\graphicspath{ {images/} }
\usepackage[font=small,labelfont=bf]{caption}
\usepackage[superscript,biblabel]{cite}
\usepackage{url}

\newenvironment{Figure}
  {\par\medskip\noindent\minipage{\linewidth}}
  {\endminipage\par\medskip}

\begin{document}

\maketitle
%\begin{multicols}{2}

%%%%%%%%%%%%%%%%%%%%%%%%%%%%%%%%%%%
%%%% ********		   PROJECT OVERVIEW          ******** %%%%%
%%%%%%%%%%%%%%%%%%%%%%%%%%%%%%%%%%%

\pagebreak
\section{Project Overview}

Sourcemash is a web-based application that categorizes news articles to help users read their RSS feeds faster. Sourcemash achieves this goal is three steps. First, users select their sources from the Browse page, which offers a pre-seeded list of popular feeds and a form allowing manual input of new RSS feeds; Sourcemash is designed to categorize your news, not someone else's. Second, Sourcemash's categorization engine analyzes each article in the user's subscribed feeds to find the categories that best match; our algorithm is constantly working in the background, so users can find articles in the topics they're looking for faster. Finally, Sourcemash offers the ability to read thew news by both category and feed, offering users the ability to integrate their news reading preferences into the application experience.

Sourcemash targets users of RSS feed aggregator subscriptions -- popular web-app Feedly currently has a market size of 15 million users (http://googlecloudplatform.blogspot.com/2014/04/feedly-delivers-seamless-experience-to-15-million-users-with-app-engine-and-cloud-storage.html) with 50,000 paying for premium services -- and aims to enhance the news reading experience by eliminating the article redundancy inherent in subscribing to similar news sources and offering a category-driven perspective to the news that reduces the time required to parse related content.

%%%%%%%%%%%%%%%%%%%%%%%%%%%%%%%%%%%
%%%% ********		  USER INSTRUCTIONS  		******** %%%%%
%%%%%%%%%%%%%%%%%%%%%%%%%%%%%%%%%%%

\section{User Instructions}

Sourcemash is a web application, located at \url{www.sourcemash.com}. The source code for Sourcemash is available at \url{https://github.com/sourcemash/Sourcemash}.

\subsection{Production}

Users are encouraged to use the production release of Sourcemash available at \url{www.sourcemash.com}. The website uses a Material Design framework\cite{Materialize} to create an easy-to-navigate, responsive user experience. The navigation menu, pinned to the left side of the page, is the primary tool used to access the site's core features and pages, which are explained in detail below:

\begin{itemize}
	\item \textbf{Browse} (\url{sourcemash.com/browse}): The Browse page contains an offering of feeds that users can subscribe to. If users do not see a feed that they wish to add, they may add their own using the form at the top of the page.
	\item \textbf{Categorizer} (\url{sourcemash.com/categorizer}): This page offers standalone access to the Sourcemash categorization engine. The categorizer allows users to input the URL of any online article and receive auto-generated categories based on our categorization algorithm.
	\item \textbf{Category or Feed View} (e.g. \url{sourcemash.com/feeds/1}): By clicking on a Category or Feed title under the tabs in the left-pinned navigation menu, users can view and interact with all items present in that category or feed.
	\item \textbf{Saved} (\url{sourcemash.com/saved}): Authenticated users can find the list of items that they have bookmarked here.
	\item \textbf{Profile} (\url{sourcemash.com/profile}): Authenticated users can manage their settings here. For example, Sourcemash offers a "Show Suggested Content" feature that will display one additional article inside a category page from a feed that a user is not subscribed to. This helps introduce users to new feeds.
\end{itemize}

\subsection{Development}

Instructions for installation, execution, and testing are maintained at \url{https://github.com/sourcemash/Sourcemash/blob/master/README.md}.


%%%%%%%%%%%%%%%%%%%%%%%%%%%%%%%%%%%
%%%% ********      TECHNICAL DESCRIPTION 	******** %%%%%
%%%%%%%%%%%%%%%%%%%%%%%%%%%%%%%%%%%

\section{Technical Description}

Sourcemash was built to be used as an RSS feed aggregator, but at its core is a categorization algorithm. We will break down the technical description into these two components: the categorization algorithm and the RSS reader that incorporates the algorithm to allow users to get parse through the news faster.

\subsection{Categorization Algorithm}

The categorization algorithm has two fundamental implementation requirements. First, because Sourcemash aims to group or cluster articles by category, it is critical that the algorithm generates categories with a sufficient number of overlapping categories between articles. Second, the algorithm needs to work quickly and efficiently in order to stay up-to-date with the latest news.

Satisfying the overlapping requirement required a strategy to normalize our category data. As an initial pass, this meant that capitalization, punctuation, and pluralization needs to be consistent for each instance of a given category. Upon researching state-of-the-art categorization techniques, we came across the CommunityCluster algorithm\cite{Grineva}, which uses Wikipedia lookups to generate possible keywords from an article. The Wikipedia API normalizes any query strings by removing any inconsistency with punctuation or capitalization.

For the efficiency requirement, we increase the speed of the CommunityCluster algorithm by pre-processing article data and running the algorithm on only the n-most-common n-grams in each feed article. This require less lookups using the Wikipedia API, which is the biggest constraint on the algorithm's runtime.

The CommunityCluster algorithm requires a heuristic to determine the relatedness between to Wikipedia articles. The metric we used relies on the number of overlapping links in a Wikipedia article. On Wikipedia, articles contain inline hyperlinks to related wikipedia articles. By calculating the percent of hyperlinks that overlap between two Wikipedia articles, we can get a sense of their similarity.

The current implementation of the categorization algorithm works as follows:

\begin{enumerate}

\item Generate a bag-o-words for article $a$, overweighting counts for ngrams from the article title and longer phrases (e.g. bigrams, trigrams).
\item From the word counts generated in the bag-o-words, select the 20 most common ngrams as the keyword candidates $C$.
\item For each keyword candidate $c \in C$:
  \begin{enumerate}
  	\item Search Wikipedia for whether $c$ exists as an article title $w$.
  	\begin{enumerate}
 		 \item If $w$ exists but is ambiguous (i.e. has a Wiki disambiguation page), map the keyword $c$ to all related Wikipedia articles linked to from the disambiguation page, denoted $W' = \{w'_{1}, w'_{2}, ...\}$.
  		\item Otherwise, map $c$ to the unambiguous Wikipedia article $w$.
  	\end{enumerate}
	\item Memoize the hyperlinks in $w$ (or in all $w' \in W'$) to quickly compute relatedness scores later.
  \end{enumerate}
\item For each $c$ that mapped to one unambiguous $w$, store $w$ in the list of assigned Wikipedia articles for the article $a$, denoted $A$.
\item For each $c'$ that mapped to an ambiguous set $W'$, disambiguate $W'$ by summing the relatedness scores between each $w' \in W'$ and all articles $w \in A$ from Step 4 and appending $w'$ with the max total relatedness score to the list of assigned articles $A$.
\item Generate the graph $G$, where each $w \in A$ is a vertex and the edge weight $|e_{ww'}|$ is the relatedness score between articles $w$ and $w'$.
\item Perform the Louvain method for community detection\cite{Blondel} to isolate the communities in $G$.
\item Extract the densest communities' keywords to be used as the categories for article $a$.
\end{enumerate}

To test the categorizer on any online article, go to \url{sourcemash.com/categorizer}.

\subsection{RSS Reader}

The site is built as a RESTful API written in Flask, a Python web microframework, for the backend, and a single-page app built in Backbone.js for the frontend. A redis server runs in the background to asynchronously schedule all emails and other worker tasks.

%%%%%%%%%%%%%%%%%%%%%%%%%%%%%%%%%%%
%%%% ********      		EVALUATION 			******** %%%%%
%%%%%%%%%%%%%%%%%%%%%%%%%%%%%%%%%%%

\section{Evaluation}

While

\subsection{Criteria}

When

\subsection{Letter of Support: Professor Devin Balkcom}

When


%%%%%%%%%%%%%%%%%%%%%%%%%%%%%%%%%%%
%%%% ********		ACKNOWLEDGEMENTS  	******** %%%%%
%%%%%%%%%%%%%%%%%%%%%%%%%%%%%%%%%%%

\subsection*{Acknowledgments}
Devin Balkcom, Professor of Computer Science at Dartmouth College, provided key guidance in the direction of Sourcemash's development throughout the duration of CS 98: Senior Design and Implementation Project. Michael Evans and Sravana Reddy, Neukom Fellows from the Neukom Institute at Dartmouth College, engaged in useful discussion with the authors and provided plentiful advice and feedback. Without their support, Sourcemash could not have achieved such success.

%\end{multicols}

\begin{thebibliography}{99}

\bibitem{Materialize}
  ``Materialize: A modern responsive front-end framework based on Material Design.'' Available at \url{http://materializecss.com/}. MIT, 2015.

\bibitem{Grineva}
  M. Grineva, M. Grinev and D. Lizorkin, ``Extracting key terms from noisy and multi-theme documents.'' \emph{Proceedings of the 18th international conference on World wide web}. ACM, 2009.

\bibitem{Blondel} Blondel, V.~D., Guillaume, J.-L., Lambiotte, R., \& Lefebvre, E.``Fast unfolding of communities in large networks.'' \emph{Journal of Statistical Mechanics: Theory and Experiment} 10(8), 2008.

\end{thebibliography}

\end{document}

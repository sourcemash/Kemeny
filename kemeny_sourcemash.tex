\documentclass[11pt]{article}
\title{\includegraphics[scale=0.65]{solologo} \\~\\ \includegraphics[scale=0.75]{logotext} \\~\\ Kemeny Prize Application 2015}
\author{
	Alex Gerstein \\ Dartmouth College '15 \\ Computer Science \\ \texttt{alexander.s.gerstein.15@dartmouth.edu}
	\\ \\
	Scott Gladstone \\ Dartmouth College '15 \\ Computer Science \\ \texttt{scott.w.gladstone.15@dartmouth.edu}
	}
\date{\today}

\usepackage[margin=1.0in]{geometry}
\setlength{\parskip}{10pt plus 1pt minus 1pt}
\usepackage{fancyhdr}
\usepackage{enumerate}
\usepackage{multicol}
\usepackage{fixltx2e}
\usepackage{graphicx}
\graphicspath{ {images/} }
\usepackage[font=small,labelfont=bf]{caption}
\usepackage[superscript,biblabel]{cite}
\usepackage{url}

\newenvironment{Figure}
  {\par\medskip\noindent\minipage{\linewidth}}
  {\endminipage\par\medskip}

\begin{document}

\maketitle
%\begin{multicols}{2}

%%%%%%%%%%%%%%%%%%%%%%%%%%%%%%%%%%%
%%%% ********		   PROJECT OVERVIEW          ******** %%%%%
%%%%%%%%%%%%%%%%%%%%%%%%%%%%%%%%%%%

\pagebreak
\section{Project Overview}

Sourcemash is a web-based RSS feed aggregator that algorithmically categorizes articles for users to browse their feed by category. Typically, RSS feed readers only allow users to look through their news feed-by-feed. By categorizing articles and regrouping them in this way, users can find the topics that interest them faster and without re-reading similar stories from different sources. Sourcemash is ideal for news junkies who want the full picture on a story in a way that's easier to digest and consumes less of their time.

%%%%%%%%%%%%%%%%%%%%%%%%%%%%%%%%%%%
%%%% ********		  USER INSTRUCTIONS  		******** %%%%%
%%%%%%%%%%%%%%%%%%%%%%%%%%%%%%%%%%%

\section{User Instructions}

Sourcemash is a web application, located at \url{www.sourcemash.com}. The source code for Sourcemash is available at \url{https://github.com/sourcemash/Sourcemash}.

\subsection{Production}

Users are encouraged to use the production release of Sourcemash available at \url{www.sourcemash.com}. The website uses a Material Design framework\cite{Materialize} to create an easy-to-navigate, responsive user experience. The navigation menu, pinned to the left side of the page, is the primary tool used to access the site's core features and pages, which are explained in detail below:

\begin{itemize}
	\item \textbf{Browse} (\url{sourcemash.com/browse}): The Browse page contains an offering of feeds that users can subscribe to. If users do not see a feed that they wish to add, they may add their own using the form at the top of the page.
	\item \textbf{Categorizer} (\url{sourcemash.com/categorizer}): This page offers standalone access to the Sourcemash categorization engine. The categorizer allows users to input the URL of any online article and receive auto-generated categories based on our categorization algorithm.
	\item \textbf{Category or Feed View} (e.g. \url{sourcemash.com/feeds/1}): By clicking on a Category or Feed title under the tabs in the left-pinned navigation menu, users can view and interact with all items present in that category or feed.
	\item \textbf{Saved} (\url{sourcemash.com/saved}): Authenticated users can find the list of items that they have bookmarked here.
	\item \textbf{Profile} (\url{sourcemash.com/profile}): Authenticated users can manage their settings here. For example, Sourcemash offers a "Show Suggested Content" feature that will display one additional article inside a category page from a feed that a user is not subscribed to. This helps introduce users to new feeds.
\end{itemize}

\subsection{Development}

Instructions for installation, execution, and testing are maintained at \url{https://github.com/sourcemash/Sourcemash/blob/master/README.md}.


%%%%%%%%%%%%%%%%%%%%%%%%%%%%%%%%%%%
%%%% ********      TECHNICAL DESCRIPTION 	******** %%%%%
%%%%%%%%%%%%%%%%%%%%%%%%%%%%%%%%%%%

\section{Technical Description}

Sourcemash was built to be used as an RSS feed aggregator, but at its core is a categorization algorithm. We will break down the technical description into these two components: the categorization algorithm and the RSS reader that incorporates the algorithm to allow users to get parse through the news faster.

\subsection{Categorization Algorithm}

The categorization algorithm has two fundamental implementation requirements. First, because Sourcemash aims to group or cluster articles by category, it is critical that the algorithm generates categories with a sufficient number of overlapping categories between articles. Second, the algorithm needs to work quickly and efficiently in order to stay up-to-date with the latest news.

Satisfying the overlapping requirement required a strategy to normalize our category data. As an initial pass, this meant that capitalization, punctuation, and pluralization needs to be consistent for each instance of a given category. Upon researching state-of-the-art categorization techniques, we came across the CommunityCluster algorithm\cite{Grineva}, which uses Wikipedia lookups to generate possible keywords from an article. The Wikipedia API normalizes any query strings by removing any inconsistency with punctuation or capitalization.

For the efficiency requirement, we increase the speed of the CommunityCluster algorithm by pre-processing article data and running the algorithm on only the n-most-common n-grams in each feed article. This require less lookups using the Wikipedia API, which is the biggest constraint on the algorithm's runtime.

The CommunityCluster algorithm requires a heuristic to determine the relatedness between to Wikipedia articles. The metric we used relies on the number of overlapping links in a Wikipedia article. On Wikipedia, articles contain inline hyperlinks to related wikipedia articles. By calculating the percent of hyperlinks that overlap between two Wikipedia articles, we can get a sense of their similarity.

The current implementation of the categorization algorithm works as follows:

\begin{enumerate}

\item Generate a bag-o-words for article $a$, overweighting counts for ngrams from the article title and longer phrases (e.g. bigrams, trigrams).
\item From the word counts generated in the bag-o-words, select the 20 most common ngrams as the keyword candidates $C$.
\item For each keyword candidate $c \in C$:
  \begin{enumerate}
  	\item Search Wikipedia for whether $c$ exists as an article title $w$.
  	\begin{enumerate}
 		 \item If $w$ exists but is ambiguous (i.e. has a Wiki disambiguation page), map the keyword $c$ to all related Wikipedia articles linked to from the disambiguation page, denoted $W' = \{w'_{1}, w'_{2}, ...\}$.
  		\item Otherwise, map $c$ to the unambiguous Wikipedia article $w$.
  	\end{enumerate}
	\item Memoize the hyperlinks in $w$ (or in all $w' \in W'$) to quickly compute relatedness scores later.
  \end{enumerate}
\item For each $c$ that mapped to one unambiguous $w$, store $w$ in the list of assigned Wikipedia articles for the article $a$, denoted $A$.
\item For each $c'$ that mapped to an ambiguous set $W'$, disambiguate $W'$ by summing the relatedness scores between each $w' \in W'$ and all articles $w \in A$ from Step 4 and appending $w'$ with the max total relatedness score to the list of assigned articles $A$.
\item Generate the graph $G$, where each $w \in A$ is a vertex and the edge weight $|e_{ww'}|$ is the relatedness score between articles $w$ and $w'$.
\item Perform the Louvain method for community detection\cite{Blondel} to isolate the communities in $G$.
\item Extract the densest communities' keywords to be used as the categories for article $a$.
\end{enumerate}

To test the categorizer on any online article, go to \url{sourcemash.com/categorizer}.

\subsection{RSS Reader}

The site is built as a RESTful API written in Flask, a Python web microframework, for the backend, and a single-page app built in Backbone.js for the frontend. A redis server runs in the background to asynchronously schedule all emails and other worker tasks.

%%%%%%%%%%%%%%%%%%%%%%%%%%%%%%%%%%%
%%%% ********      		EVALUATION 			******** %%%%%
%%%%%%%%%%%%%%%%%%%%%%%%%%%%%%%%%%%

\section{Evaluation}

While other solutions exist to group and categorize related articles, only Sourcemash allows users to choose their own sources. Sourcemash is the first and only RSS feed reader that categorizes at the article-level to reorganize your news. This unique feature makes it stand out from competing RSS feed readers, such as Feedly.

\subsection{Accuracy}
While the categorization algorithm often finds accurate ``top'' categories for an article, it faces some issues with adding poor additional categorizations from close communities. For example, for a CNN article titled ``Deadliest outbreak of Ebola virus: What you need to know''\cite{Ebola}, the accurate categories ``Ebola'', ``Virus'', ``Peace Crops'', ``Liberia'' and ``Sierra Leone'' are returned along with innacurate categories ``Gary Sick''and ``Tornado outbreak''. These poor categorizations likely derive from incorrectly disambiguated versions of words such as ``sick'' or ``natural disaster'' that are present in the article. 

As observed by Grineva, Grinev and Lizorkin\cite{Grineva}, the community-clustering approach is approximately 60-70\% effective at eliminating completely unrelated categories, but inaccurately disambiguated words sometimes pass the relatedness metric. When this occurs, Sourcemash post-processes generated categories to hide poor results from the user. For example, Sourcemash only displays categories that have a sufficient number of matching articles; this helps remove the rarer, often incorrectly selected, categories. With Sourcemash, we have found a similar accuracy ratio in selected categories as Grineva's team (e.g. 5 of 8 [62.5\%] of the categories assigned to the CNN article\cite{Ebola} referenced above were considered to be ``accurate portrayals of the article'' by surveyed individuals). 

To maintain a level of specificity in the categories, we also ignore any categories that are ``nested'' or contained within other categories. For example, the categorization engine would assign ``Google Maps'' as a category over ``Google,'' if both are present in an article's set of assigned categories.


\subsection{Letter of Support: Professor Devin Balkcom}

When


%%%%%%%%%%%%%%%%%%%%%%%%%%%%%%%%%%%
%%%% ********		ACKNOWLEDGEMENTS  	******** %%%%%
%%%%%%%%%%%%%%%%%%%%%%%%%%%%%%%%%%%

\subsection*{Acknowledgments}
Devin Balkcom, Professor of Computer Science at Dartmouth College, provided key guidance in the direction of Sourcemash's development throughout the duration of CS 98: Senior Design and Implementation Project. Michael Evans and Sravana Reddy, Neukom Fellows from the Neukom Institute at Dartmouth College, engaged in useful discussion with the authors and provided plentiful advice and feedback. Without their support, Sourcemash could not have achieved such success.

%\end{multicols}

\begin{thebibliography}{99}

\bibitem{Materialize}
  ``Materialize: A modern responsive front-end framework based on Material Design.'' Available at \url{http://materializecss.com/}. MIT, 2015.

\bibitem{Grineva}
  M. Grineva, M. Grinev and D. Lizorkin, ``Extracting key terms from noisy and multi-theme documents.'' \emph{Proceedings of the 18th international conference on World wide web}. ACM, 2009.

\bibitem{Blondel} Blondel, V.~D., Guillaume, J.-L., Lambiotte, R., \& Lefebvre, E.``Fast unfolding of communities in large networks.'' \emph{Journal of Statistical Mechanics: Theory and Experiment} 10(8), 2008.

\bibitem{Ebola} http://www.cnn.com/2014/03/27/world/ebola-virus-explainer/

\end{thebibliography}

\end{document}
